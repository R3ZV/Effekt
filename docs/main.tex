\documentclass[14pt]{extarticle}

% --------------------------------------------------
% Packages
% --------------------------------------------------
\usepackage[utf8]{inputenc}
\usepackage[T1]{fontenc}
\usepackage{amsmath, amssymb, amsfonts}
\usepackage{graphicx}
\usepackage{geometry}
\usepackage{enumitem}
\usepackage{float}
\usepackage[linesnumbered,ruled,vlined]{algorithm2e}
\usepackage{listings}
\usepackage{xcolor}
\usepackage{caption}
\usepackage{hyperref}
\usepackage{cleveref}
\usepackage{parskip} % For better paragraph spacing

\geometry{margin=2cm}

% --------------------------------------------------
% Hyperref setup
% --------------------------------------------------
\hypersetup{
    colorlinks=true,
    linkcolor=blue,
    urlcolor=blue,
    citecolor=blue,
    pdftitle={Implementation of Different Filters}
}

% --------------------------------------------------
% Listings setup
% --------------------------------------------------
\lstset{
    backgroundcolor=\color{gray!10},
    basicstyle=\ttfamily\footnotesize,
    frame=single,
    numbers=left,
    numberstyle=\tiny,
    breaklines=true,
    tabsize=2,
    keywordstyle=\color{blue},
    commentstyle=\color{green!50!black}
}

% --------------------------------------------------
% Title
% --------------------------------------------------
\title{Effekt -- Implementation of Different Filters}
\author{Lucan Cristian and Maracine Razvan}
\date{\today}

% --------------------------------------------------
% Document
% --------------------------------------------------
\begin{document}

\maketitle
\tableofcontents
\newpage

\section{State Variable Filter (SVF)}

\begin{figure}[H]
    \centering
    \includegraphics[width=0.7\linewidth]{Effekt/SVF.png}
    \caption{SVF Discrete-Time Model (ZDF Topology)}
    \label{fig:SVF_diagram}
\end{figure}

\subsection{Description}

The State Variable Filter (SVF) is a highly versatile topology that simultaneously computes three primary filter outputs: low-pass ($y_{LP}$), high-pass ($y_{HP}$), and band-pass ($y_{BP}$). 

Based on the Zero-Delay Feedback (ZDF) structure, the band-pass output is derived as:
\[
y_{BP} = g(x - 2Ry_{BP} - gy_{BP} - s_2) + s_1
\]
Solving for $y_{BP}$ yields the stable implementation formula:
\[
y_{BP} = \frac{g(x - s_2) + s_1}{1 + g(g + 2R)}
\]
From this, the low-pass and high-pass outputs are extracted:
\begin{align*}
y_{LP} &= g \cdot y_{BP} + s_2 \\
y_{HP} &= x - 2R \cdot y_{BP} - y_{LP}
\end{align*}

This filter utilizes two trapezoidal integrators, represented as $g\xi + s_i$, where:
\begin{itemize}
    \item $g$: The zero-delay feedback gain.
    \item $\xi$: The instantaneous input of the integrator.
    \item $s_1, s_2$: The internal states (memory) of the integrators.
\end{itemize}

The parameter $g$ represents the cutoff frequency after \textbf{frequency prewarping}, which maps analog frequencies ($0$ to $\infty$) to digital frequencies ($0$ to Nyquist) using:
\[
g = \tan\left(\frac{\text{cutoff} \cdot \pi}{2}\right)
\]

\subsection{Results: Chester's Scream Analysis}

Below are the spectrograms for the outputs of the SVF filter applied to the vocals of Chester Bennington from Linkin Park's \href{https://www.youtube.com/watch?v=Gd9OhYroLN0}{"Crawling"} (20s--34s).

\begin{figure}[H]
    \centering
    \includegraphics[width=0.48\linewidth]{Effekt/band_pass_crawling.png}
    \caption{Band-pass output (Cutoff: 0.15, Resonance: 0.5)}
    \label{fig:svf_band_pass_ex}
\end{figure}

\begin{figure}[H]
    \centering
    \includegraphics[width=0.48\linewidth]{Effekt/low_pass_crawling.png}
    \caption{Low-pass output (Cutoff: 0.15, Resonance: 0.5)}
    \label{fig:svf_low_pass_ex}
\end{figure}

\begin{figure}[H]
    \centering
    \includegraphics[width=0.48\linewidth]{Effekt/high_pass_crawling.png}
    \caption{High-pass output (Cutoff: 0.15, Resonance: 0.5)}
    \label{fig:svf_high_pass_ex}
\end{figure}

\subsection{Resonance and Damping}

The variable $R$ represents the \textbf{damping factor} (the mathematical inverse of resonance). It controls the amount of band-pass signal fed back into the summing junction. While the cutoff controls the frequency boundary, $R$ controls the amplitude peak at that boundary.

\begin{figure}[H]
    \centering
    \includegraphics[width=0.48\linewidth]{Effekt/low_pass_crawling_low_res.png}
    \caption{Low-pass (Resonance: 0.01)}
    \label{fig:low_pass_crawling_low_res}
\end{figure}

\begin{figure}[H]
    \centering
    \includegraphics[width=0.48\linewidth]{Effekt/low_pass_crawling_super_high_res.png}
    \caption{Low-pass (Resonance: 0.99) -- Note the resonant peak.}
    \label{fig:low_pass_crawling_super_high_res}
\end{figure}

When resonance is high, an intense line appears at the cutoff frequency.

\subsection{Secondary Filter Types}

By linear combination of the primary SVF outputs, we can derive specialized filters.

\subsubsection*{Band-Shelving Filter}
Using a gain parameter $K$, we can boost or cut the band-pass region:
\[ y_{S} = x + 2RK \cdot y_{BP} \]
\begin{figure}[H]
    \centering
    \includegraphics[width=0.5\linewidth]{Effekt/band_shelving_crawling.png}
    \caption{Band-shelving (Cutoff: 0.15, Res: 0.99, K: 0.5)}
    \label{fig:shelving_ex}
\end{figure}

\subsubsection*{Notch Filter}
A notch filter is achieved by setting $K = -1$, effectively subtracting the resonant peak from the input:
\[ y_{N} = x - 2R \cdot y_{BP} \]
\begin{figure}[H]
    \centering
    \includegraphics[width=0.5\linewidth]{Effekt/notch_crawling.png}
    \caption{Notch filter (Cutoff: 0.15, Res: 0.99)}
    \label{fig:notch_ex}
\end{figure}

\subsubsection*{All-Pass Filter}
Setting $K = -2$ creates an all-pass response, affecting only phase:
\[ y_{AP} = x - 4R \cdot y_{BP} \]
\begin{figure}[H]
    \centering
    \includegraphics[width=0.5\linewidth]{Effekt/all_pass_crawling.png}
    \caption{All-pass filter (Cutoff: 0.15, Res: 0.99)}
    \label{fig:all_pass_ex}
\end{figure}

\subsubsection*{Peaking Filter}
The peaking (or phase-flip) filter is derived from the difference of the low-pass and high-pass signals:
\[ y_{P} = y_{LP} - y_{HP} \]
\begin{figure}[H]
    \centering
    \includegraphics[width=0.5\linewidth]{Effekt/peaking_crawling.png}
    \caption{Peaking filter output}
    \label{fig:peaking_ex}
\end{figure}


\end{document}